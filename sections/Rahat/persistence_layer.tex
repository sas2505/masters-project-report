\section{Persistence Layer Migration [Rahat Rafiq]}\label{sec:persistence_layer_migration}

When developing the dvdstore spring boot application an embedded H2 database was used as a persistence layer. However, H2 is an in-memory database which means the data is only persistent as long as the application is running. For both staging and production, a better production-grade database server is needed for better load balancing and data persistence. The AKS CI-CD solution on azure cloud utilizes a PostgreSQL server as a persistence layer. The PostgreSQL database is configured as deployment in the Kubernetes cluster. To create the deployment in the cluster a docker containerized version of the PostgreSQL server is needed, this image can be obtained from the PostgreSQL official docker hub page. The PostgreSQL deployment is configured to create a database and a user by utilizing a previously created Kubernetes configmap. The database layer is made persistent by creating a persistent volume for the PostgreSQL deployment in the cluster. A persistent volume claim is mounted at the correct mount-path of the PostgreSQL service pods. This persistent volume will create permanent storage space on the node machine for storing PostgreSQL data.      