
\subsection{Scope of testing in Continuous Integration / Continuous Deployment}\label{sec:Scope of Continuous Integration / Continuous Deployment}

\textit{Author: Wakeel Khan}

\subsubsection{Scope of testing in Continuous Integration(CI):}

Let's say we have multiple developers working on the same application each of them
has their own feature or bug fix that they are working on and they are all contributing
by moving their code into the same code repository to the same application. The code
repository could be a version control system like GitLab, when a developer pushes a code
to the repository it is linked to a build management system like Jenkins for instance
that takes the code and builds it if your application is based on java then you can think
of this as a jar file since there are multiple developers pushing their code to the same
application. We want to make sure it builds properly and does not introduce any new
problems into the application, so we can configure the build system to run tests on the
build package through testing framework which can test your application Api's or web
interfaces. Once we have successfully passed the tests, we can confirm that the build cycle
is complete. This whole process of providing the development and working environments
to multiple developers so they can work simultaneously on the same product at same time
without effecting the working and problem-solving process of others is called continuous
integration. \par

Continuous integration overcome the problem of conflicts error between developers, best
approach is that developer needs to push their work on the repository on daily basis. This
scenario is useful in the cases when the developers are making frequent commits because
here it is easier to resolve the problems and conflicts." \cite{larman2004agile}


\subsubsection{Scope of testing in Continuous Deployment(CD):}

Continuous Delivery (CD) is the method of as quickly as possible bringing new build into
the hands of user. It is the next step after CI and is an approach used to minimize the
risks involved in releasing software application and its new functions. Releasing software
updates is very painful and time-taken process. Continuous deployment eliminates the
costs and efforts associated with this process by ensuring that any alteration made to an
application's underlying code is releasable, which ensures that each upgrade is smaller and
can be delivered more easily to users. Teams can make their development process more
effective, less costly, and can get input from users faster by making releases less drastic
events that can be done on demand whenever new code is available. By simply rolling
out the next update, if bugs are found in development, they can be squashed easily. \par


According to a researcher, many companies have adopted continuous deployment because
it provides emphasize upon rapid feedback, recurring deployment for the sake of customer
satisfaction as well as enhancing the quality and productivity." \cite{c12c3fdcbeb64c018fb3b6dadcd2c936} 

\subsubsection{Role of Automated Testing in CI/CD Pipeline:}


When the latest piece of software has been developed, it must be developed and then
carefully reviewed to ensure that it satisfies all the initial business requirements. There
are different testing methodologies that can be used to ensure that an application works
properly and provide desired results, covering everything from functional tests to performance
tests. \par

According to a researcher Today's CI/CD pipeline is a collection of many technologies
that ensures the highest quality product reaches production". \cite{wolf2016automated} \par

The main objective of Continuous Integration/Continuous Delivery (CI/CD) pipeline
development is to gather all the teams(development, testing, operations) involved in
Information Technology (IT) project on same level and streamline the process of the
automated software release. Automation is the foremost objective. \par


Automated provisioning of the environment allows teams with only a few clicks to handle
test environments. Even the latest versions of browsers, applications, and resolution
settings can include the necessary automated testing tool, ensuring Quality Assurance
(QA) teams will avoid having to spin up, manage, or fully tear down environments.
Anything from unit test to system test and even environmental provisioning in the software
development cycle should be automated. 

