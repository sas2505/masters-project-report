

\subsection{Test Driven Development}\label{sec:Test Driven Development}

\textit{Author: Keyur Panchal}

Test-Driven Development, or Test Driven Development (TDD), is a software development discipline in which developers write automated test cases for improvements or new features before writing code. The basic premise of TDD is that you first write an error test for the simplest function that you need to implement. Then write the simplest code possible to pass this test once it is complete. The new code will be rewarded or refactored
if it meets the code requirement standards. \par

This software development strategy has continued to gain increased attention as one of the core extreme programming practices." \cite{1510569}

\subsubsection{Advantage of TDD}

Advantage of TDD is, for example TDD helps ensure effciency by concentrating on specifications before writing code. By breaking it down into tiny, practical actions, it helps keep simple code easy and testable. It offers documentation of how everybody interacts in the system. When it later joins the team, it builds itself and acts as a catalyst for quick change and repeatable regression testing.

\subsubsection{Disadvantage of TDD}

Major disadvantage is that the test cases are heavy in maintenance because we are already creating a test case. For instance, we have created 50 test cases and developed a functionality now, the test case is failing or something bad is happening then maintenance cost is high in that case if the requirement of changing, test case will change maintenance will change, development effort will change, so, that is a disadvantage in TDD.


\subsubsection{Steps for Achieving TDD}

To perform the TDD they are like sequence of steps that you must do.
\begin{itemize}
\item First, is that you will read the requirement.  
\item Add the test cases of it.  
\item Run the test cases if it will fail.  
\item Then next step is to write the code to fix it.  
\item And then Run the tests again if they are passed or not you will refactor your code and again you will repeat the same process until your old test cases will pass and your functionality will be completed.  
\end{itemize}




