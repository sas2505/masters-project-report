

\subsection{Load Testing}\label{sec:Load Testing}

\textit{Author: Keyur Panchal}

\subsubsection{Why Do We Need Performance Testing?}

Performance testing is nothing but testing the stability and responsiveness of an application
by applying a load. This is known as performance testing. When you develop
any software when developers will develop any software, customers or business analysts
or product managers they will tell whatever we are developing it should be able to handle
how many users at a time. Stability is nothing but ability to withstand design number of
users that is called as stability.

Many large-scale software systems must service thousands or millions of concurrent requests.
These systems must be load tested to ensure that they can function correctly
under load (i.e., the rate of the incoming requests)." \cite{7123673}

The response time is nothing, but time taken to send the request to the server run that
program in the server and then finally taking back the request from the server it is called
as response time. For example, whenever you deal with any application and you are trying
to do a payment so when you are on the payment page and you are doing the payments
you have entered all your credit card details or debit card details and you're clicking on the
send button what actually happens, the request will go to the server it will run a program
in the server and finally it will send the request from the server which will change the
page from payments page to the another page. How it happened, how much time it is
taking that is called the Response Time.


Now what is load, load is nothing but number of users using the application at a particular period that is called as load.

\subsubsection{What is Load Testing?}

Load test is a very common term in the testing one some people or client also refers to performance testing as load testing. The term load has its specific meaning, and it is a type of performance test or you can also call this non-functional test now come to the definition of load test. Load test is conducted to check the performance of an application under peak load. The load test we tried to identify whether a system or an application is stable enough to handle the peak load for a certain duration we concentrate on the application states like response time user handling capacity Data Processing System (DPS) handling capacity of the application along with resource usage like memory usage, Central Processing Unit (CPU) usage, etc.


\textbf{Concept}
The common bottleneck which are identified in the load test is that our application fails to handle the peak load application takes a long time to response high CPU and memory utilization you may get some more specific performance bug if the application is not tuned enough. Now try to understand what peak load is. Peak load is the highest load which is identified during a day or a month or a year and it depends on the data set which you have taken from the previous year data. For example, on Black Friday sale, many e-commerce websites go down because of very much user load on these websites. So, to tackle this problem we must perform a load test before and make sure that our website performs normal in peak conditions as well.


\subsubsection{Different Tools For Load Testing}

\textbf{JMETER}
\begin{itemize}
\item \textbf{Created by:} Apache Foundation
\item \textbf{Written in:} Java
\item \textbf{License:} Apache 2.0
\item \textbf{Scriptable:} Limited(XML)
\end{itemize}
\textbf{K6}
\begin{itemize}
\item \textbf{Created by:} Load Impact
\item \textbf{Written in:} Go
\item \textbf{License:} AGPL3
\item \textbf{Scriptable:} Yes: JS

\end{itemize}





