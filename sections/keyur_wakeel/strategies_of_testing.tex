


\subsection{Strategies of Testing}\label{sec:Strategies of Testing}

\textit{Author: Keyur Panchal}

\subsection{Manual Testing:}
Manual testing is performed on the basis of some cases which are written manually by some human tester. Tester runs the software on these test cases to make sure that the application is running properly without any errors. This means that someone actually goes on a device to evaluate numerous components of the application including that the business and technical requirements are also being fulfilled. There are cases when it is not possible to perform automation testing so in these scenarios manual testing comes into play. Let's say if you want to test a mobile application under unpredictable conditions you can use manual testing like when mobile application is running and it goes in the hands of children and they make numerous taps on the touch screen of mobile phone then how the application will react in this unexpected situation and will it go on functioning properly without any errors. Well, you can't apply automation there right, so that's when manual testing becomes handy. Basically, the entire testing process is executed by a human health.

\textbf{Advantages:}
\begin{itemize}
\item Manual testing allows the tester to check application under various harsh conditions.
When the application is deployed or goes live then the underlying hidden problems
in the application also goes live and these errors can be easily traced using manual
testing, this type of testing is called life testing.
\item In Manual testing less programming knowledge is required. Basically, when you're
testing an application manually the main focus is on understanding the requirements,
documenting the test case and executing test cases as you are performing
them as live so your focus on programming is usually less.
\item Manual testing is also helpful to find problems that effect the usability of application.
When the final product goes in the hands of user and any error that may arise when user utilizes the application in a specific way, these errors are also easily tracked
using manual testing. So in case of visual testing manual testing is really helpful.
\item Manual testing is cheap in terms of investment and required skill set. Manual testing
does not require deep expertise in the field of IT and development and a regular
user can also be involved in the process of testing. Additionally, it is also feasible
for short frequent testing scenarios because of its cost effectiveness.
\item In the development phase of application, there are frequent changes and additions
to the application's piece of code and so these changes are also needed to be testing
immediately and frequently and in this scenario too, manual testing is handy.
\end{itemize}


\subsubsection{Automation Testing:}
Automation testing focuses on automating the testing process of any application or software piece of code. It removes the involvement of manual human testing with devices or applications and thus increasing the effectiveness of overall testing process. It uses test cases written as scripts or different testing tools to test the product creating artificial environments that are harsh as compared to the normal working of application. As compared to manual testing this type of testing is much more reliable and yields faster results. Additionally, it also makes easier to carry out different test types like regression and performance testing or load testing by simulating high load usage scenario.

\textbf{Advantages of Automation Testing:}
\begin{itemize}
\item Simplifies test cases execution.
\item Improves Reliability of tests.
\item Increases speed of test executions.
\item Saves time and money.
\item Reduces maintenance cost of testing.
\item Improves accuracy of software tests.
\item Increase amount of test coverage.
\item Minimizing human interaction with testing.
\end{itemize}

\subsubsection{Comparison between Automation Testing and Manual testing:}

\textbf{Manual}
\begin{itemize}
\item Manual test is carried out by human tester in a manual way.
\item It takes time to test same thing again and again.
\item More human and electrical resources required.
\item Manual testing is handy when testing the addition of new features.
\item Creating multiple testing environments to test the application in same test cases is time taking.
\end{itemize}

\textbf{Automation}
\begin{itemize}
\item Automation testing is done with the help of a tool and script.
\item Less time is taken carry out same test using automation scripts but script writing requires expertise.
\item Less human resources but more skill sets required.
\item Automation testing is used to perform regression testing.
\item Multiple machines providing multiple testing environments can perform same test cases.
\end{itemize}



