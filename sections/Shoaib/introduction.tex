
\subsection{Motivation [Shoaib Bin Anwar]}
CI/CD means continuous integration and continuous delivery or deployment. This process is triggered when a code is committed into the version control system. A build tool is used on the CI/CD to create the artifacts. The next step is to run some unit and integration tests to find out any bugs. Afterward, after passing all the tests, the new version of the software is deployed on the desired environment. Otherwise, the developers will work on the bugs and fixing those until it is on deployable states. In this way, when the final release date will come, the software can be released smoothly and faster.

CI/CD can be set up on a cloud environment that incorporates the main process of CI/CD (build, test, delivery, and release) and provides some features flexibility for the CI/CD. One of the main upgrades is that the software development and release team does not need to worry about the environment specification and different tools set up in that environment.

In software releases, there are some manual processes used. Some processes are manually configured the software dependency on the production environment, deploying the software on the environment. These processes are more complicated due to sparse environment-related issues and so on. Furthermore, these processes often indicate missing bugs in the software, which are found in the phase that is rather risky to fix. And doing this rush also introduces more bugs into the system, and team members must work on a lot of stress and pressure in the end. But if these processes are made automated in the first phases, these issues can be reduced a lot initially. And thus, CI/CD becomes a growing and interesting concept in the software release process. It is integrated from the beginning, so whenever the developer commits, the build is done. Then test cases are executed, and based on the results, the decision will be taken whether this version will be released. If it was not built or failed, the developer should work on the fix, and afterward, the fix will be committed. The CI will do the process again, and then upon the success of the CI, it will be deployed on the desired environment as well. This automated deployment in the environment is called CD. Using these two methodologies, CI/CD makes the system smoother, bug-free, and stress-free software release process.
Furthermore, using CI/CD on the cloud gives additional edges for the team regarding software release steps. Especially, setting up the environment and scaling the projects are more flexible and easier. In this report, we will dive more into the CI/CD steps and different tool structures.

\subsection{Project Report Structure [Shoaib Bin Anwar]}
This report is divided into several segments. In the next couple of chapters, we will discuss the basics of CI/CD, discussion about different tools of CI/CD, Infrastructure as Code, security, and privacy of CI/CD. After that, in the next part, we will discuss the CI/CD implementations in AWS and Azure. Following that, we will finally discuss the challenges and also the cost and performance analysis.
