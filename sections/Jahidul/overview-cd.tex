\subsection{Continuous Deployment/Delivery in cloud-native architecture [Md Jahidul Haque]}

Continuous deployment in SDLC involves the automation of the software release process via configuration scripts in deployment automation technologies. Github offers Github Actions feature which is the main platform for describing the continuous deployment process. Gihub offers developers to develop their release process via Github's own domain specific language (DSL). Github's DSL supports {{.yml}} scripts to write github workflows. Github Workflow enables various jobs for deployment inside the development environment. Developers can manage their repository and create automated tasks to serialize the batch processes during the deployment. The main purpose of this continuous deployment is initiate the continuous integration smoothly without the manual configuration of the repository. After conducting all the acceptance testing the automation jobs can also create new dependencies via the {{.yml}} files. But the continuous deployment lacks the infrastructure automation which brings the manual configuration of the public cloud infrastructure. This technological need brings the concept of continuous delivery.

Continuous delivery refers to deployment automation when developer is able to automate the cloud infrastructure via the infrastructure automation objects. This enables the developer to manage,provision and scale the whole infrastructure which is needed for their developed software. In the infrastructure automation process, Github Action also involves for managing the pipeline where services in AWS are responsible for creating the infrastructure objects. The main technology of continuous delivery is CloudFormation from AWS. CloudForamtion is a complete infrastructure automation tool for continuous delivery of a software. 

In summary, the CD pipeline of the SDLC means the automation of the release processes which are related with automatic deployment environment, network configuration and infrastructure object creation. For automating deployment environment, Github actions can be used to describe the environment definitions to create resources that are needed for the application. Then other batch jobs execute deployment environment and test the application via acceptance testing. After successful deployment, the infrastructure objects creation are defined via CloudFomation's  cloud development kit(CDK). In this CDK, we can define Java classes for describing the infrstrructure objects and then that objects can be compiled via AWS's CLI. All the infrastucture automation commands is included in the Github actions so that infrastructure objects can automatically be created via {{.yml}} scripts.

In this report, we briefly define the our implementation of AWS based cloud-native
pipeline automation. In the starting stage of the pipeline we integrated Continuous integration which is described in section ~\ref{sec:jahidul_sec_1} and then we implemented Continuous deployment/delivery described in this section.  


