\section{Continuous Integration in Cloud Native architecture [Md Jahidul Haque]}\label{sec:jahidul_sec_1}

Software releasing process can be procedural in the conventional software development life-cycle(SDLC). In traditional SDLC, the software development mostly depends on the different development teams working on the different parts of the same project. Ideally, developers need to collaborate between themselves to mitigate the knowledge gaps to ensure a successful release of the software. During the software releasing process, the code review of any changes in the existing software is very crucial in SDLC. After reviewing the code by the code review team, the new changes in the code is inspected but he testing team for the quality assurance of the new changes. This new changes is then passes through various unit testing and  integration testing by the testing team. Furthermore, the QA passed patch of the software is handed over the deployment team who are responsible for the releasing the new version of the software. This involves DevOps tasks for the deployment team. The DevOps engineers create and manage different cloud services to provision and scale the software deployment during the release. After Provisioning and scaling the new version of the software is released. 

The cloud-native architecture suggest software release via public clouds like AWS, Google Cloud or the Microsoft Azure. Also, this approach overcome the various hurdles of setting up on-premise infrastructure. Yet, the public cloud based software development  suffers vendor lock-in issues. In the traditional way of software release described in the previous paragraph, managing cloud-native application is much tougher to provision and scale manually for the deployment team. This creates time lagging in SDLC for any new modification of the software. In this particular case, CI/CD comes in practice among the software development community. CI/CD stands for continuous integration and continuous deployment/delivery. In this novel approach, SDLC can achieve more granular gain from the resource optimization during the release process of the software. Moreover, this process helps developers to automate tests and make QA process more free from human errors. 

The Continuous integration of the cloud-native application means the automation of the software releasing process via automation of various SDLC processes. For example, the automation in the repository management in continuous integration reefers to development of automation scripts that can automatically handle any new changes of the repository of a application code-base. These automation scripts can be handled different technologies like Gihub Actions, GitLab CI/CD, Jenkins, CircleCI. For public cloud like the AWS or the Azure, Github or GitLab can be used by the developers to developer automation scripts for managing the repository in the case of modification or new release of the software.

In this report we will talk about the implementation of the automation of the software release for the spring-boot application. At the beginning of the implementation we have a spring-boot application which have several services exposed and the application is using Gradle package management for dependency management. In our implementation, we will implement continuous by developing a CI pipeline for integrating the application in AWS cloud services. Our implementation involves two major parts: continuous integration and continuous deployment/delivery. 







