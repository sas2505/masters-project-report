
\section{Introduction [Md Jahidul Haque]}
Software  releasing  process  can  be  procedural  in  the  conventional  software  development life-cycle(SDLC). In traditional SDLC, the software development mostly depends on the different development teams working on the different parts of the same project.  Ideally,developers  need  to  collaborate  between  themselves  to  mitigate  the  knowledge  gaps  to ensure a successful release of the software.  During the software releasing process, the code review of any changes in the existing software is very crucial in SDLC. After reviewing the code by the code review team, the new changes in the code is inspected but he testing team for the quality assurance of the new changes.  This new changes is then passes through various unit testing and integration testing by the testing team.  Furthermore,  the QA passed  patch  of  the  software  is  handed  over  the  deployment  team  who  are  responsible for  the  releasing  the  new  version  of  the  software.   This  involves  DevOps  tasks  for  the deployment team.  The DevOps engineers create and manage different cloud services to provision and scale the software deployment during the release.  After Provisioning and scaling the new version of the software is released.

The cloud-native architecture suggest software release via public clouds like AWS, Google Cloud or the Microsoft Azure. Also, this approach overcome the various hurdles of setting up on-premise infrastructure. Yet, the public cloud based software development  suffers vendor lock-in issues. In the traditional way of software release described in the previous paragraph, managing cloud-native application is much tougher to provision and scale manually for the deployment team. This creates time lagging in SDLC for any new modification of the software. In this particular case, CI/CD comes in practice among the software development community. CI/CD stands for continuous integration and continuous deployment/delivery. In this novel approach, SDLC can achieve more granular gain from the resource optimization during the release process of the software. Moreover, this process helps developers to automate tests and make QA process more free from human errors. 


The Continuous integration of the cloud-native application means the automation of the software releasing process via automation of various SDLC processes. For example, the automation in the repository management in continuous integration reefers to development of automation scripts that can automatically handle any new changes of the repository of a application code-base. These automation scripts can be handled different technologies like Gihub Actions, GitLab CI/CD, Jenkins, CircleCI. For public cloud like the AWS or the Azure, Github or GitLab can be used by the developers to developer automation scripts for managing the repository in the case of modification or new release of the software.

Continuous deployment in SDLC involves the automation of the software release process via configuration scripts in deployment automation technologies. Github offers Github Actions feature which is the main platform for describing the continuous deployment process. Gihub offers developers to develop their release process via Github's own domain specific language (DSL). Github's DSL supports {{.yml}} scripts to write github workflows. Github Workflow enables various jobs for deployment inside the development environment. Developers can manage their repository and create automated tasks to serialize the batch processes during the deployment. The main purpose of this continuous deployment is initiate the continuous integration smoothly without the manual configuration of the repository. After conducting all the acceptance testing the automation jobs can also create new dependencies via the {{.yml}} files. But the continuous deployment lacks the infrastructure automation which brings the manual configuration of the public cloud infrastructure. This technological need brings the concept of continuous delivery.

Continuous delivery refers to deployment automation when developer is able to automate the cloud infrastructure via the infrastructure automation objects. This enables the developer to manage,provision and scale the whole infrastructure which is needed for their developed software. In the infrastructure automation process, Github Action also involves for managing the pipeline where services in AWS are responsible for creating the infrastructure objects. The main technology of continuous delivery is CloudFormation from AWS. CloudForamtion is a complete infrastructure automation tool for continuous delivery of a software. 

In this report we will talk about the implementation of the automation of the software release for the spring-boot application. At the beginning of the implementation we have a spring-boot application which have several services exposed and the application is using Gradle package management for dependency management. In our implementation, we will implement continuous by developing a CI pipeline for integrating the application in AWS and Azure cloud services. Our implementation involves two major parts: continuous integration (CI) and continuous deployment/delivery. For both the cases of CI and CD, we will discuss two different pipeline implementation: one for Azure and another one for AWS cloud. 


\subsection{Motivation [Shoaib Bin Anwar]}
CI/CD means continuous integration and continuous delivery or deployment. This process is triggered when a code is committed into the version control system. A build tool is used on the CI/CD to create the artifacts. The next step is to run some unit and integration tests to find out any bugs. Afterward, after passing all the tests, the new version of the software is deployed on the desired environment. Otherwise, the developers will work on the bugs and fixing those until it is on deployable states. In this way, when the final release date will come, the software can be released smoothly and faster.

CI/CD can be set up on a cloud environment that incorporates the main process of CI/CD (build, test, delivery, and release) and provides some features flexibility for the CI/CD. One of the main upgrades is that the software development and release team does not need to worry about the environment specification and different tools set up in that environment.

In software releases, there are some manual processes used. Some processes are manually configured the software dependency on the production environment, deploying the software on the environment. These processes are more complicated due to sparse environment-related issues and so on. Furthermore, these processes often indicate missing bugs in the software, which are found in the phase that is rather risky to fix. And doing this rush also introduces more bugs into the system, and team members must work on a lot of stress and pressure in the end. But if these processes are made automated in the first phases, these issues can be reduced a lot initially. And thus, CI/CD becomes a growing and interesting concept in the software release process. It is integrated from the beginning, so whenever the developer commits, the build is done. Then test cases are executed, and based on the results, the decision will be taken whether this version will be released. If it was not built or failed, the developer should work on the fix, and afterward, the fix will be committed. The CI will do the process again, and then upon the success of the CI, it will be deployed on the desired environment as well. This automated deployment in the environment is called CD. Using these two methodologies, CI/CD makes the system smoother, bug-free, and stress-free software release process.
Furthermore, using CI/CD on the cloud gives additional edges for the team regarding software release steps. Especially, setting up the environment and scaling the projects are more flexible and easier. In this report, we will dive more into the CI/CD steps and different tool structures.

\subsection{Project Report Structure [Shoaib Bin Anwar]}
This report is divided into several segments. In the next couple of chapters, we will discuss the basics of CI/CD, discussion about different tools of CI/CD, Infrastructure as Code, security, and privacy of CI/CD. After that, in the next part, we will discuss the CI/CD implementations in AWS and Azure. Following that, we will finally discuss the challenges and also the cost and performance analysis.
